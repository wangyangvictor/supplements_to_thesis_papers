\documentclass[12pt]{article}

\usepackage{amsmath,amssymb,amsthm,bm,colonequals,microtype,xcolor}
\usepackage{vywang}
\usepackage{hyperref}

% \title{Thesis outline}
\title{Thesis outline, errata, and comments}
% \thanks{The link \href{https://tinyurl.com/hooley33}{\nolinkurl{tinyurl.com/hooley33}} will always point to the latest version of this document.}}
% \thanks{See \href{https://tinyurl.com/hooley33}{\nolinkurl{tinyurl.com/hooley33}} for the latest version of this document.}}
% \author{Victor Wang}
\author{Victor Wang}
% \thanks{\emph{Email address}: \href{mailto:vywang@math.princeton.edu}{\nolinkurl{vywang@math.princeton.edu}}}}
% \date{}
\date{\today}

\begin{document}

\maketitle
% \tableofcontents

\section{Thesis outline}

(See \href{https://wangyangvictor.github.io/thesis_links.html}{\nolinkurl{wangyangvictor.github.io/thesis_links.html}} for individual links.
Questions, comments, corrections, and suggestions are all welcome.)

% Let $F(\bm{x})\colonequals x_1^3+x_2^3+\cdots+x_6^3$.
% Let $N(X)\colonequals \#\{\bm{x}\in [-X,X]^6: x_1^3+x_2^3+\cdots+x_6^3 = 0\}$.
For $n,X\in \mathbb{Z}_{\geq0}$,
let $r_3(n)\colonequals \#\{x,y,z\in \mathbb{Z}_{\geq0}: x^3+y^3+z^3=n\}$ and $M_2(X)\colonequals \sum_{a\leq X^3}r_3(a)^2$.
% Hua (1938) proved $N_F(X)\ll_\epsilon X^{7/2+\epsilon}$ unconditionally;
% lowering $7/2$ remains an open problem today.
Conditionally on Langlands-type hypotheses and GRH (for certain Hasse--Weil $L$-functions),
Hooley (1997) and Heath-Brown (1998) proved $M_2(X)\ll_\epsilon X^{3+\epsilon}$.
% up to $\epsilon$, this is the best possible bound.
Furthermore,
Hooley (1986) conjectured $M_2(X)\sim c_{\textnormal{HLH}}X^3$ (as $X\to\infty$) for a specific constant $c_{\textnormal{HLH}}\in \mathbb{R}_{>0}$,
which is \emph{strictly greater} than the Hardy--Littlewood constant $c_{\textnormal{HL}}\in \mathbb{R}_{>0}$.

My thesis consists of three parts:
\begin{enumerate}
    \item Paper~I:
    \emph{Diagonal cubic forms and the large sieve}
    (42 pages).
    
    This shows that Hooley's (and Heath-Brown's) hypotheses can be replaced with
    a large-sieve hypothesis a la Bombieri--Vinogradov.
    
    \item Paper~II:
    \emph{Isolating special solutions in the delta method:
    The case of a diagonal cubic equation in evenly many variables over $\mathbb{Q}$}
    (34 pages).
    
    Heath-Brown's work, and morally also Hooley's work, is based on the ``delta method'' for $M_2(X)$.
    One can easily ``extract'' $c_{\textnormal{HL}}X^3$ from the delta method.
    Paper~II extracts $(c_{\textnormal{HLH}}-c_{\textnormal{HL}})X^3$ in a natural way.
    
    \item Paper~III:
    \emph{Approaching cubic Diophantine statistics via mean-value $L$-function conjectures of Random Matrix Theory type}
    (136 pages).
    
    Building on Paper~II,
    we prove (i) a general localized form of Hooley's conjecture
    and (ii) that asymptotically $100\%$ of integers $a\not\equiv \pm4\bmod{9}$ are sums of three cubes,
    conditionally on some standard number theory conjectures---the main additions (relative to Hooley and Heath-Brown) being conjectures of Random Matrix Theory and Square-free Sieve type.
    To reduce (i) to these conjectures,
    we introduce several new \emph{unconditional} ingredients.
    % Paper~III is long because there are many distinct ``sources of $\epsilon$'' in the works of Hooley and Heath-Brown.
    For example, certain complete exponential sums ``fail square-root cancellation'' quite badly---and thus do not fall under ``standard'' conjectural frameworks---and we prove new results that help to control such behavior.
\end{enumerate}

(Thanks to Nick Katz for helpful suggestions on wording.)

\section{Papers~I--III: Errata and comments}

For now, all errata and comments refer to the versions of August 7, 2021:
\begin{enumerate}
    \item \href{https://arxiv.org/abs/2108.03395v1}{\texttt{arXiv:2108.03395v1}} for I,
    
    \item \href{https://arxiv.org/abs/2108.03396v1}{\texttt{arXiv:2108.03396v1}} for II,
    and
    
    \item \href{https://arxiv.org/abs/2108.03398v1}{\texttt{arXiv:2108.03398v1}} for III.
\end{enumerate}
Thanks below are given in parentheses; \textcolor{gray}{resolved issues} (fixed in later drafts, either released or to be released) are grayed out.

\subsection{Paper~I (\href{https://arxiv.org/abs/2108.03395v1}{\texttt{arXiv:2108.03395v1}})}

\begin{itemize}
    \item Paragraph after Definition~I.1.3:
    \textcolor{gray}{It would be good to state Vaughan's record, of the form $N_F(X)\ll X^{7/2}/(\log{X})^c$.}
    (Wooley)
    
    \item Right after Definition~I.1.5:
    \textcolor{gray}{Replace ``[Hoo86] observed several properties of the sums $S_{\bm{c}}(n)$,'' with ``The sums $S_{\bm{c}}(n)$ have some nice properties''.}
    (As was helpfully pointed out to me by Wooley, the properties are not \emph{due} to [Hoo86], but were rather \emph{employed} by [Hoo86] to great effect.)
    
    \item Paragraph before Definition~I.1.11:
    \textcolor{gray}{This is OK,
    but it would be more canonical to say ``if $\bm{c}\in \ZZ^m$ and $p\nmid F^\vee(\bm{c})$, then $(\mcal{V}_{\bm{c}})_{\FF_p}$ is a smooth complete intersection in $\PP^{m-1}_{\FF_p}$ of dimension $m_\ast$ and multi-degree $(3,1)$''.}
    
    \item Definition~I.1.11:
    \textcolor{gray}{Replace ``$\dim H^{m_\ast}_{\map{prim}}(V_{\bm{c}}\times\CC)$'' (which should have been ``$\rank H^{m_\ast}_{\map{prim}}(V_{\bm{c}}\times\CC)$'')
    with ``$\rank(H^{m_\ast}_\textnormal{sing}(V_{\bm{c}}(\CC),\ZZ)/H^{m_\ast}_\textnormal{sing}(\PP^{m-1}(\CC),\ZZ))$''.}
    
    Then \textcolor{gray}{delete ``where $H^{m_\ast}_{\map{prim}}$\dots Betti\dots (\emph{primitive}\dots)''.}
    % https://mathoverflow.net/questions/186378/when-did-betti-cohomology-come-to-be-used-the-way-it-is-today-and-how-is-it
    
    \item After Definition~I.1.11:
    % Add a remark that ``in Definition~I.1.11, we use the embedding $V_{\bm{c}}\belongs \PP^{m-1}_\QQ$ to define $H^{m_\ast}_\textnormal{sing}(V_{\bm{c}}(\CC),\ZZ)/H^{m_\ast}_\textnormal{sing}(\PP^{m-1}(\CC),\ZZ)$''.
    \textcolor{gray}{Add a remark that
    (i) ``for each $\bm{c}$ above,
    $V_{\bm{c}}$ is a \emph{subvariety} of $\PP^{m-1}_\QQ$ by definition,
    so $H^{m_\ast}_\textnormal{sing}(V_{\bm{c}}(\CC),\ZZ)/H^{m_\ast}_\textnormal{sing}(\PP^{m-1}(\CC),\ZZ)$ is well-defined'';
    and (ii) ``each $L_p(s,V_{\bm{c}})$ above is well-defined:
    for any $\bm{c},\bm{c}'\in \ZZ^m$ with $V_{\bm{c}}=V_{\bm{c}'}$,
    one can show that $E_{\bm{c}}(q) = E_{\bm{c}'}(q)$ holds for all prime powers $q$ coprime to $F^\vee(\bm{c})F^\vee(\bm{c}')$''.}
    % Fancy proof in (ii): smooth proper base change; elementary proof: compare homogeneous ideals to show that $\bm{c}\in \QQ^\times \bm{c}'$, and then use the coprimality condition to show that $\bm{c}\in \ZZ_p^\times \bm{c}'$.
    
    (To avoid discussing (i)--(ii),
    we could write $-\bm{1}_{2\mid m_\ast} + \rank H^{m_\ast}_\textnormal{sing}(V_{\bm{c}}(\CC),\ZZ)$ in place of $\rank(\cdots/\cdots)$,
    and $L_p(s,\bm{c})$ in place of $L_p(s,V_{\bm{c}})$.
    But the current notation is more transparent and suggestive.)
    
    \item \S{I.4}:
    \textcolor{gray}{Since $F$ is diagonal, it might be more efficient to quote results from Vaughan's textbook (or old work of Hua) rather than [Hoo88, Hoo14]} (Wooley),
    though \textcolor{gray}{a little care is needed since Vaughan focuses on Waring's problem}.
    Furthermore, \textcolor{gray}{the case $m\in \set{4,6}$ (and essentially $m=5$ too) can be quoted} from [HB98].
    
    % \item On interpolation being the only known method for Hua bound:
    % understand Vaughan's work better... (read Vaughan 1986 and 2020).
\end{itemize}

\subsection{Paper~II (\href{https://arxiv.org/abs/2108.03396v1}{\texttt{arXiv:2108.03396v1}})}

\begin{itemize}
    \item Remark~II.1.20 and \S{II.5.2}:
    Some (purely expository) comments are missing obvious hypotheses.
    In 1.20, $I_{\bm{c}}(n)$ is only ``morally positive'' if $w\geq 0$.
    In \S5.2, some of the comments only apply if $\sigma_{\infty,L^\perp,w}\neq 0$ (and in particular, $L\cap (\Supp{w})\neq \emptyset$).
    
    \item \S{II.5.2}:
    Some of the $n$'s should be $q$'s.
\end{itemize}

\subsection{Paper~III (\href{https://arxiv.org/abs/2108.03398v1}{\texttt{arXiv:2108.03398v1}})}

\begin{itemize}
    \item Paragraph after Remark~III.1.9:
    Remove ``essentially''.
    % from ``essentially suggests''.
    (Sarnak)
    
    \item Definition~III.3.8:
    Replace ``Also let'' with ``And for each prime $p$, let''.
    Write ``$\map{diff}$'' instead of ``$\map{prim}$'',
    to avoid conflict with the usual definition of $H^\bullet_{\map{prim}}$.
    Also (for convenience), generalize ``$H^d(\PP^{1+d})$'' and ``\emph{hypersurface} $W/\QQ$ of dimension $d\geq 0$''
    to ``$H^d(\PP^{r+d})$'' and ``\emph{complete intersection} $W/\QQ$ of dimension $d\geq 0$ and codimension $r\geq 1$''.
    
    Then modify the following accordingly:
    Remark~3.10,
    Definition~3.11,
    Conjecture~3.18,
    Observation~3.21,
    paragraph after Observation~4.2,
    point~(2) on p.~73 (before Remark~A.4),
    and Remark~A.11(2).
    % remove ``M_c\cong'' and say ``(which is isomorphic to M_c)''
    
    \item Definition~III.3.11:
    Add ``And for all $p,j$, let $\tilde{\alpha}_{\bm{c},j}(p)\defeq \tilde{\alpha}_{M,j}(p)$, where $M\defeq H^{m_\ast}_{\map{prim}}(V_{\bm{c}})$''.
    
    \item Conjecture~III.3.13~(EKL):
    ``$\bm{c}+r\wh{\ZZ}$'' should be ``$\bm{c}+\map{rad}(n)r\wh{\ZZ}$''.
    (Cf.~\S1.2, hypothesis~(4), ``modulus $O(p)\cdot \gcd(D(\bm{c})^{O(1)}, p^\infty)$''.)
    
    With this typo corrected, our \emph{applications} of (EKL)---all contained in \S\S7.2--7.3 and \S\S7.8--7.9---all remain correct as written.
    
    \item Right after Remark~III.3.42:
    Add a remark that we expect that with a lot of additional technical work, one could remove the condition $(\Supp{w})\cap(\map{hess}{F})_\RR=\emptyset$ in Theorem~3.39(b).
    Cf.~Remark~4.9.
    
    \item \S{III.4.1}:
    Replace ``Definition~3.8'' (both times) with ``Definition~3.11''.
    
    \item \S{III.7.5}:
    In the statement and derivation of 7.25~(RA1'), some factors of $\abs{t}^\eps$ are missing (since the ``GRH bound'' is missing a $(1+\abs{t})^\eps$).
    In the statement of 7.25~(RA1') and derivation of 7.26~(RA1'E),
    replace ``$n_0\leq Z^\hbar$'' (both times) with ``$n_0\leq Z^{\hbar^2}$'',
    and ``$n_0\geq Z^\hbar$'' with ``$n_0\geq Z^{\hbar^2}$''.
    Then in the derivation of (RA1'),
    explicitly choose $\hbar\ls 1$ small
    so that in item~(2) in the second paragraph, $(1+\abs{t})^\eps n_0^{O(1)}\leq Z^{\hbar/2}$.
    
    (These changes are important for \S10, but not for \S9.)
    
    One could also remark that \emph{morally}, when applying (RA1) here (towards (RA1')) and elsewhere (via (RA1'), (RA1'E), and (RA1'E')),
    it suffices to work with boxes $\mcal{B}(\bm{Z})$ of \emph{intermediate} ``lopsidedness'' (say $\leq Z/Z^{1-\hbar} = Z^\hbar$),
    and with moduli $n_0$ and shifts $\abs{t}$ of \emph{small} size (say $\leq Z^{\hbar^2}$).
    % (The other ranges can be handled without (RA1).)
    
    \item \S{III.7.9.3}, derivation of (RA1’E’L):
    Explicitly cite (EKL) to justify the existence of all the $\wh{\ZZ}$-averages under consideration,
    and furthermore to justify the ``constancy of $a'_{\bm{c}}(n_\star)$'' in observation~(2) on p.~57.
    
    \item Definition~III.C.2:
    \textcolor{gray}{Add ``projective'' before ``variety $Y/k$'',
    so that Remark~C.3 holds as written.}
    % More generally, one could instead replace $H^\bullet$ (in C.2 and C.3) with $H_{c}^\bullet$.
    % Also, add ``embedded'' before ``projective variety $X\belongs \PP^n_k$''.
    (Katz)
    
    % \item Definitions~III.C.2 and~III.C.4, and proof of Observation~III.C.6:
    % Replace ``error-relevant'' with ``non-planar'' (more accurate and transparent).
    
    % \item Definition~III.C.4:
    % For robustness,
    % append ``, and hypersurface $Y\belongs \PP^n_{\FF_q}$ isomorphic to $X$ (as an abstract variety)'' to ``for every prime $\ell\nmid q$'',
    % and replace ``$H^\bullet(X)$'' with ``$H^\bullet(Y)$''.
    
    % Then in C.4 and C.6, replace ``hypersurface $X\belongs \PP^n_{\FF_q}$'' with ``variety $X/\FF_q$ isomorphic to a hypersurface in $\PP^n_{\FF_q}$''.
    % And adjust the proof of C.6 accordingly.
    
    Also, the current notion of ``error-relevant'' is extrinsic (Katz),
    would be better termed ``non-planar'',
    and is not symmetric enough for us (given that $H^i(\PP^n_k)\to H^i(X)$, for $i\leq 2\dim{X}$, could presumably fail to be injective for some embedded projective variety $X\belongs \PP^n_k$).
    To resolve these issues robustly and cleanly,
    \begin{itemize}
        \item \textcolor{gray}{\emph{append} ``; and for each $i\geq 0$, let $\mcal{E}^i(Y)$ denote the multiset of (geometric) Frobenius eigenvalues on $H^i(Y)$'' to the \emph{second} sentence of C.2,}
        and
        \item \textcolor{gray}{\emph{replace} the \emph{third} with
        ``Now for a projective variety $X/k$ of dimension $N\geq 0$,
        let $\mcal{E}^i_{\triangle}(X,\PP)\defeq (\mcal{E}^i(X)\cup \mcal{E}^i(\PP^N_k))\setminus (\mcal{E}^i(X)\cap \mcal{E}^i(\PP^N_k))$ for $i\geq 0$
        (so that if $\alpha\in \ol{\QQ}_\ell$ has multiplicities $j_1,j_2\geq 0$ in $\mcal{E}^i(X),\mcal{E}^i(\PP^N_k)$, then it has multiplicity $\abs{j_1-j_2}$ in $\mcal{E}^i_{\triangle}(X,\PP)$),
        and let $\mcal{E}_{\triangle}(X,\PP)\defeq \bigsqcup_{i\geq 0}\mcal{E}^i_{\triangle}(X,\PP)$.''}
    \end{itemize}
    Then \textcolor{gray}{right after C.3, add the following theorem (and proof sketch).}
    Below, we will refer to the theorem as ``Theorem~P''.
    \theoremstyle{plain}
    \newtheorem*{theorem*}{Theorem}
    \begin{theorem*}
    [Deligne, Katz, Skorobogatov, and Ghorpade--Lachaud]
    \label{THM:general-perversity-result}
    Let $k\defeq \FF_q$.
    Fix integers $n,N\geq 1$,
    and a complete intersection $X\belongs \PP^n_k$ with $\dim{X} = N$ and $\codim{X}\geq 1$.
    Let $D\defeq \dim(\map{Sing}(X_{\ol{k}}))$, with the convention $\dim(\emptyset)\defeq -1$.
    Then for $i\in \ZZ$, the following hold.
    \begin{enumerate}
        \item If $i\geq N+D+2$, then $\mcal{E}^i_{\triangle}(X,\PP) = \emptyset$.
        
        \item If $i = N+D+1$, then $\mcal{E}^i(\PP^{N}_k)\belongs \mcal{E}^i(X)$.
    \end{enumerate}
    \end{theorem*}
    
    % \rmk{
    % By [\href{https://arxiv.org/abs/0808.2169v1}{\texttt{arXiv:0808.2169v1}}, Proposition~3.2], one could add the following to Theorem~P:
    % (3) if $i = N$, then $\mcal{E}^i(\PP^{N}_k)\belongs \mcal{E}^i(X)$;
    % and (4) if $0\leq i\leq N-1$, then $\mcal{E}^i_{\triangle}(X,\PP) = \emptyset$.
    % But we only need (1)--(2).
    % }
    
    % \rmk{
    % We only need Theorem~P when $X$ is a hyperplane section of $V$,
    % in which case the result follows directly from \cite{skorobogatov1992exponential}*{Corollary~2.2}.
    % But for the reader's convenience, we have stated the result generally.
    % }
    
    \pf{
    [Proof sketch]
    Claim~(1) follows from [Hoo91b, Katz's Appendix, assertion~(2) in the proof of Theorem~1].
    And if $D = -1$, then (2) follows from weak Lefschetz.
    Now assume $D\geq 0$, and let $i\defeq N+D+1$.
    If $2\nmid i$, then $\mcal{E}^i(\PP^{N}_k) = \emptyset$, so (2) holds trivially.
    Now suppose $2\mid i$.
    Then $\mcal{E}^i(\PP^{N}_k) = \set{q^{i/2}}$, since $i\leq 2N$ by generic smoothness.
    
    It remains to show that $q^{i/2}\in \mcal{E}^i(X)$;
    [\href{https://arxiv.org/abs/0808.2169v1}{\texttt{arXiv:0808.2169v1}}, first sentence of Remark~3.5] essentially states this without proof,\footnote{though (1), [Poo17, Corollary~7.5.21], and [\texttt{arXiv:0808.2169v1}, Theorem~2.4 or Skorobogatov (1992), after a Veronese embedding]
    might allow for a proof by induction on $\codim{X}$,
    % (using the fact that $\dim\Sing$ decreases by at most $1$ for a hyperplane section (so that $\codim\Sing$ is weakly decreasing under taking hyperplane sections),
    % and the fact that projective complete intersections with $\codim\Sing \geq 2$ are always irreducible;
    % and in the case where $\codim\Sing = 1$, we can directly use [Poo17, Corollary~7.5.21] instead of inducting),
    which the authors may have had in mind}
    so it seems appropriate to sketch one.
    Let $T\defeq \Aff^1_k$.
    Following Katz (essentially), we can reduce to the case in which there exists
    a closed subscheme $Z\belongs \PP^n_T$, flat over $T$, such that (i) $Z_0 = X$ and (ii) $Y\defeq Z_1$ is a smooth complete intersection in $\PP^n_k$ with $\dim{Y} = N$.
    % Note: Clearly $Z/T$ is proper (since $Z$ is closed in $\PP^n_T$).
    In this case, [SGA~7~I, Deligne's Expos\'{e}~I, Corollaire~4.3] implies that the specialization map $H^i(Z_t)\to H^i(Z\times_{T} \ol{k(T)}, \QQ_\ell)$ is an isomorphism at $t=1$ (since $D\geq 0$), and a surjection at $t=0$.
    By $G_k$-equivariance, it follows that $\mcal{E}^i(Y)\belongs \mcal{E}^i(X)$.
    But $i\geq N+1$ (since $D\geq 0$),
    so $\mcal{E}^i(Y) = \mcal{E}^i(\PP^{N}_k)$ (by (1) for $Y$).
    % (by weak Lefschetz and Poincar\'{e} duality).
    Thus $\set{q^{i/2}} = \mcal{E}^i(\PP^{N}_k)\belongs \mcal{E}^i(X)$.
    }
    
    Then do the following:
    \begin{itemize}
        \item \textcolor{gray}{In C.4,
        replace ``all of the error-relevant Frobenius eigenvalues on $H^\bullet(X)$'' with ``all $\alpha\in \mcal{E}_{\triangle}(X,\PP)$''.}
    
        \item \textcolor{gray}{In C.4 and C.6,
        generalize ``projective hypersurface $X\belongs \PP^n_{\FF_q}$'' to ``projective complete intersection $X\belongs \PP^n_{\FF_q}$ with $\codim{X}\geq 1$''.
        % (for convenience).
        }
        
        \item \textcolor{gray}{In C.6:
        Convert ``$\dim(\map{Sing}(X_{\ol{k}}))=0$'' to ``$\dim(\map{Sing}(X_{\ol{k}}))\leq 0$''.}
        
        (Then \textcolor{gray}{make corresponding changes elsewhere.})
        
        \item \textcolor{gray}{In C.6:
        To be safe, replace the ``$18(3+\deg{X})^{n+1}$'' in (1) with ``$18(3+\codim{X}\deg{X})^{n+1}2^{\codim{X}}$''.}
        
        (Then \textcolor{gray}{in Lemma~III.4.11, proof of (1),
        replace ``$18(3+3)^{m-1}$'' with ``$72(3+6)^m$'';
        in Problem~4.16,
        ``$18(3+k)^s$'' with ``$72(3+2k)^s$'';
        in 4.16 and 4.19,
        ``$M_{d,m-1}$'' with ``$M_{d,m}$''.})
        
        \item \textcolor{gray}{In C.6 and its proof:
        Replace ``$\PP^{n-1}(k_?)$'' (all three times) with ``$\PP^{\dim{X}}(k_?)$''.}
        
        \item In C.6, proof that (2) implies (1):
        \textcolor{gray}{Replace ``$(n,1,\deg{X})$'' with ``$(n,\codim{X},\norm{\bm{d}}_\infty)$,
        if $X$ has multi-degree $\bm{d}$;
        here $\norm{\bm{d}}_\infty\leq \prod_{i}d_i = \deg{X}$''.}
        
        And \textcolor{gray}{explicitly state the LTF and Betti bounds involved.
        % And explicitly state that
        % \mathd{
        % \abs{\#X(k_r)-\#\PP^{\dim{X}}(k_r)}\leq \card{k_r}^{(\dim{X})/2}
        % \cdot \sum_{i}\sum_{Y\in \set{X, \PP^{\dim{X}}}}\dim H^i(Y);
        % }
        % and if $Y\in \set{X, \PP^N}$, then $\sum_{i}\dim H^i(Y)\leq 9(3+\codim{X}\deg{X})^{n+1}2^{\codim{X}}$.
        }
        
        \item In C.6, proof of equivalence of (2)--(3):
        \textcolor{gray}{Replace ``the only error-relevant eigenvalues can come from\dots by the usual'' with ``the multiset $\set{\alpha\in \mcal{E}_{\triangle}(X,\PP): \map{weight}(\alpha)\geq 1+\dim{X}}$
        is a sub-multiset of $\mcal{E}^{1+\dim{X}}(X)\setminus \mcal{E}^{1+\dim{X}}(\PP^{\dim{X}}_k)$ (by Theorem~P and Deligne's theory of weights). The equivalence of (2)--(3) now follows, by the usual''.}
        (Katz)
        
        And \textcolor{gray}{replace ``If $\dim{X}=1$\dots $\dim{X}\geq2$\dots \emph{with $\dim(\map{Sing}(X_{\ol{k}}))=0$}, so''
        with ``Now assume $\dim{X}\geq 1$. Then the hypothesis $\dim(\map{Sing}(X_{\ol{k}}))\leq 0$ implies that''.}
        
        \item At the end of C.6:
        \textcolor{gray}{Add the sentence ``Furthermore, (1)--(3) hold \emph{if} $\dim H^{1+\dim{X}}(X) = \dim H^{1+\dim{X}}(\PP^{\dim{X}}_k)$.''
        For proof, use Theorem~P.}
        
        Then in Appendix~C.1.1, proof of Proposition~C.9(2):
        \textcolor{gray}{Replace ``$H^i(V_{\bm{c}})/H^i(\PP^{m-1}_k)\neq0$'' (both times)
        with ``$\dim H^{1+m_\ast}(V_{\bm{c}})\neq \dim H^{1+m_\ast}(\PP^{m_\ast}_k)$'',
        and justify this (the first time) using the new ``final sentence'' of C.6.}
        
        And similarly, in Appendix~C.1.2, proof of Proposition~C.9(1):
        \textcolor{gray}{Make similar changes, appealing to [Poo17, Corollary~7.5.21] and the new ``final sentence'' of C.6.}
        
        \item In C.7:
        % Clarify that $V_{\bm{c}}$ (viewed as a subvariety of $\PP^{m-1}$) has a canonical realization as a projective hypersurface (up to equivalence),
        % depending only on $V_{\bm{c}}$.
        % (This is because if $V(F,\bm{c}\cdot\bm{x}) = V(F,\bm{c}'\cdot\bm{x})$, then $[\bm{c}] = [\bm{c}']$.)
        % Alternatively, just fix $\bm{c}$ and clarify that $\bm{c}$ determines a realization of $V_{\bm{c}}$ as a projective hypersurface (up to equivalence).
        \textcolor{gray}{Replace ``after viewing $V_{\bm{c}}$ as a projective hypersurface\dots'' with ``since $V_{\bm{c}}$ is a complete intersection in $\PP^{m-1}_k$''.}
    \end{itemize}
    
    \item Sentence before Appendix~III.C.3.1:
    \textcolor{gray}{Replace ``as shown in the proof of Observation~C.6''
    with ``by Theorem~P(1) and [Poo17, Corollary~7.5.21]''.}
    
    \item Proposition~III.C.13, proof of second part (giving an alternative approach to Proposition~C.9(2) when $m=6$):
    \textcolor{gray}{The $H^2$'s should be $H^4$'s,
    and we should consider all eigenvalues on $H^4(X)$} to be safe (in case $H^4(\PP^4_k)\to H^4(X)$ fails to be injective).
    
    Then \textcolor{gray}{one should replace each of the three expressions $\tilde{\alpha}_1^?+\dots+\tilde{\alpha}_b^?$ with $\tilde{\alpha}_1^?+\dots+\tilde{\alpha}_b^?-1$.
    The Dirichlet argument now gives $b=1$.}
    One then needs to \textcolor{gray}{prove that $\tilde{\alpha}_1 = 1$;
    this can be done by appealing to Theorem~P(2).}
    (Alternatively, one might try using the $G_\QQ$-invariance of the multiset $\set{\tilde{\alpha}_1,\dots,\tilde{\alpha}_b}$ to obtain $\tilde{\alpha}_1\in \QQ$, i.e.~$\tilde{\alpha}_1^2 = 1$.
    But without P(2), it seems hard to rule out the possibility that $\tilde{\alpha}_1 = -1$;
    taking $r\equiv 1\bmod{2}$ large does not seem to help here.)
    
    \textcolor{gray}{Some of the references to C.6 should also be replaced by appropriate references to Theorem~P(1).}
\end{itemize}

\end{document}
